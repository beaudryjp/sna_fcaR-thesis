\documentclass[../main.tex]{subfiles}

\newglossaryentry{afc}
{
    name=fca,
    text={FCA},
    description={Análisis de Conceptos Formales}
}

\newglossaryentry{ipa}
{
    name=api,
    text={API},
    description={Interfaz de Programación de Aplicaciones}
}

\newglossaryentry{agile}
{
    name=agile,
    text={agile},
    description={Enfoque iterativo a la gestión de proyectos y desarrollo de software}
}

\newglossaryentry{scrum}
{
    name=scrum,
    text={scrum},
    description={Metodología Ágil que facilita el desarrollo software}
}

\newglossaryentry{etc}
{
    name=etc,
    text={ETC},
    description={Proceso de Extracción, Transformación y Carga}
}

\newglossaryentry{sql}
{
    name=sql,
    text={SQL},
    description={Lenguaje de Consulta Estructurado}
}

\newglossaryentry{cran}
{
    name=cran,
    text={CRAN},
    description={Repositorio de paquetes R}
}

\newglossaryentry{json}
{
    name=json,
    text={JSON},
    description={Formato estándar para la representación de datos basado en la sintaxis de JavaScript}
}

\newglossaryentry{subreddit}
{
    name=subreddit,
    text={subreddit},
    plural={subreddits},
    description={Foro en la cual se publican temas}
}

\newglossaryentry{get}
{
    name=get,
    text={GET},
    description={Método de petición HTTP en la cual se solicita una representación de un recurso específico}
}

\newglossaryentry{lru}
{
    name=url,
    text={URL},
    description={Localizador de Recursos Uniforme}
}

\newglossaryentry{shiny}
{
    name=shiny,
    text={shiny},
    description={Paquete de R para crear páginas web interactivas}
}

\newglossaryentry{ui}
{
    name=ui,
    text={UI},
    description={Interfaz de Usuario}
}

\newglossaryentry{apriori}
{
    name=apriori,
    text={apriori},
    description={Algoritmo para la minería de datos para la búsqueda de un conjunto de elementos frecuentes}
}

\newglossaryentry{lift}
{
    name=lift,
    text={lift},
    description={Medida la cual indica el grado de importancia de una regla}
}

\newglossaryentry{support}
{
    name=support,
    text={support},
    description={Medida que indica que tan frecuente aparece un elemento en un conjunto de datos}
}

\newglossaryentry{docker}
{
    name=docker,
    text={docker},
    description={Plataforma de virtualización que permite crear, probar y desplegar aplicaciones rápidamente }
}

\newglossaryentry{reddit}
{
    name=reddit,
    text={reddit},
    description={Red social}
}

\newglossaryentry{rest}
{
    name=rest,
    text={rest},
    description={Interfaz de programación de aplicaciones que permite interactuar con micro-servicios que confirman el estilo REST}
}

\newglossaryentry{github}
{
    name=github,
    text={github},
    description={Proveedor que ofrece una aplicación online para el control de versiones de un proyecto}
}

\newglossaryentry{mysql}
{
    name=mysql,
    text={MySQL},
    description={Servidor de base de datos}
}

\newglossaryentry{mariadb}
{
    name=mariadb,
    text={MariaDB},
    description={Servidor de base de datos basado en MySQL}
}

\newglossaryentry{phpmyadmin}
{
    name=phpmyadmin,
    text={phpMyAdmin},
    description={Gestor de base de datos}
}

\newglossaryentry{powershell}
{
    name=powershell,
    text={PowerShell},
    description={Consola del sistema operativo Windows}
}

\newglossaryentry{csv}
{
    name=csv,
    text={CSV},
    description={Fichero que contiene valores separados por comas}
}

\newglossaryentry{latex}
{
    name=látex,
    text={látex},
    description={Sistema de composición de textos que ofrece una alta calidad tipográfica}
}

\newglossaryentry{php}
{
    name=php,
    text={PHP},
    description={Lenguaje de programación orientado al desarrollo web}
}

\newglossaryentry{rlang}
{
    name=r,
    text={R},
    description={Lenguaje de programación orientado al análisis estadístico}
}

\newglossaryentry{hasse}
{
    name=diagrma hasse,
    text={diagrama de Hasse},
    description={Representación gráfica simplificada de un conjunto parcialmente ordenado finito}
}

\newglossaryentry{dockerfile}
{
    name=dockerfile,
    text={dockerfile},
    description={Fichero de configuración para una imagen modificada para un contenedor}
}

\newglossaryentry{clang}
{
    name=c,
    text={C},
    description={Lenguaje de programación de propósito general}
}

\newglossaryentry{cpluspluslang}
{
    name=c++,
    text={C++},
    description={Lenguaje de programación de propósito general basado en C}
}

\newglossaryentry{pythonlang}
{
    name=python,
    text={python},
    description={Lenguaje de programación interpretado}
}

\newglossaryentry{ggplot}
{
    name=ggplot2,
    text={ggplot2},
    description={Paquete de R para la creación de gráficos atractivos}
}

\newglossaryentry{lattice}
{
    name=lattice,
    text={lattice},
    description={Paquete de R para la creación de gráficos de grafos}
}

\newglossaryentry{rstudio}
{
    name=rstudio,
    text={RStudio},
    description={Entorno de desarrollo para R}
}

\newglossaryentry{overleaf}
{
    name=overleaf,
    text={overleaf},
    description={Herramienta online para la escritura de documentos en látex}
}

\newglossaryentry{intent}
{
    name=intent,
    text={intent},
    description={Conjunto de objetos que posee todos los atributos en el conjunto}
}

\newglossaryentry{extent}
{
    name=extent,
    text={extent},
    description={Conjunto con los atributos comunes}
}

\newglossaryentry{iso}
{
    name=iso,
    text={iso},
    description={Organización Internacional de Normalización}
}

\newglossaryentry{http}
{
    name=http,
    text={HTTP},
    description={Protocolo usado para la transferencia de datos en Internet}
}