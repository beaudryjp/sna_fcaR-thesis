\documentclass[../../main.tex]{subfiles}

\begin{document}

Un requisito funcional puede abarcar desde la declaración abstracta de alto nivel de los requisitos de la aplicación hasta especificaciones detalladas de requisitos funcionales matemáticos. Para cada requisito funcional se establece un código formado especificado de la siguiente forma: \textbf{FR-XX}, donde la parte correspondiente a \textit{XX} referencia a un código numérico formado por dos números.

\begin{itemize}
    \item \textbf{FR-01: Obtención de datos mediante la \gls{ipa} }
    \begin{itemize}
        \item \textit{El sistema debe ofrecer la posibilidad de obtener nuevos datos referentes a las publicaciones mediante la \gls{ipa} que ofrece la red social}.
    \end{itemize}
    \item \textbf{FR-02: Normalización de datos }
    \begin{itemize}
        \item \textit{Se debe seguir una normalización de los datos para que estos tengan el formato adecuado y sean consistentes}.
    \end{itemize}
    \item \textbf{FR-03: Integración de los datos }
    \begin{itemize}
        \item \textit{El sistema integrará los datos obtenidos en el servidor de base de datos para que estos puedan ser usados posteriormente}.
    \end{itemize}
    \item \textbf{FR-04: Lectura de los datos del servidor de base de datos }
    \begin{itemize}
        \item \textit{Será posible realizar consultas personalizadas por ciertos parámetros para la obtención de los datos localizados en el servidor de base de datos}.
    \end{itemize}
    \item \textbf{FR-05: Análisis de los datos usando diversas técnicas }
    \begin{itemize}
        \item \textit{El sistema realizará varios tipos de análisis para la extracción de conocimiento de un conjunto de datos, entre las cuales se aplicará el método \gls{afc} para identificar esta información oculta}.
    \end{itemize}
    \item \textbf{FR-06: Visualización de gráficos }
    \begin{itemize}
        \item \textit{Una vez realizado el análisis de los datos se imprimirá por pantalla gráficos en los cuales se mostrará de forma visual la información más relevante del conjunto de datos}.
    \end{itemize}
    \item \textbf{FR-07: Selección de datos para la búsqueda de información }
    \begin{itemize}
        \item \textit{El sistema permitirá al usuario seleccionar y definir ciertos parámetros para la búsqueda de información en la aplicación}.
    \end{itemize}
    \item \textbf{FR-08: Exportación de datos}
    \begin{itemize}
        \item \textit{El sistema permitirá la exportación de datos en diferentes apartados de la aplicación}.
    \end{itemize}
\end{itemize}

\end{document}