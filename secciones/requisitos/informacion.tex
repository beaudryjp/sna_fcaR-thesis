\documentclass[../../main.tex]{subfiles}

\begin{document}

Los requisitos de información sirven para determinar estructurar e identificar los datos. Para cada requisito de información se establece un código formado especificado de la siguiente forma: \textbf{FIR-XX}, donde la parte correspondiente a \textit{XX} referencia a un código numérico formado por dos números. \\

\begin{itemize}
    \item \textbf{FIR-01: Subreddit}
    \begin{itemize}
        \item \textit{Identificador Subreddit, Nombre, Número aproximado de subscriptores}.
    \end{itemize}
    \item \textbf{FIR-02: Usuario}
    \begin{itemize}
        \item \textit{Identificador Usuario, Nombre, Descripción pública, Fecha Creación, Fecha añadida, Karma total, Karma de los comentarios, EsEmpleado, EsModerador, EsOro}.
    \end{itemize}
    \item \textbf{FIR-03: Publicación}
    \begin{itemize}
        \item \textit{Identificador Publicación, Título, Descripción, Relación de Votos, Número de votos, Número total de recompensas, Puntuación, Fecha Creación, Fecha añadida, Enlace Publicación, Enlace Descripción, Dominio, Identificador Subreddit, Identificador Usuario}.
    \end{itemize}
    \item \textbf{FIR-04: Moneda}
    \begin{itemize}
        \item \textit{Identificador Moneda, Nombre, Descripción, Precio Moneda, Precio Recompensa}.
    \end{itemize}
    \item \textbf{FIR-05: Comentario}
    \begin{itemize}
        \item \textit{Identificador Publicación, Identificador Usuario, Estructura, Fecha Comentario, Puntuación, Comentario}.
    \end{itemize}
    \item \textbf{FIR-06: Recompensa}
    \begin{itemize}
        \item \textit{Identificador Publicación, Identificador Moneda, Cantidad}.
    \end{itemize}
\end{itemize}




\end{document}