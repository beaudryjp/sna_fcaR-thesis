\documentclass[../../main.tex]{subfiles}

\begin{document}

\gls{rlang}\cite{doc16} es un lenguaje de programación pensado para la computación estadística y la generación de gráficos.  \\

\gls{rlang} ofrece una gran variedad de herramientas (paquetes) y técnicas, las cuales permiten realizar cualquier tipo de análisis estadístico, hacer uso de algoritmos de inteligencia artificial o realizar análisis de datos en conjuntos de datos.  \\

Los puntos fuertes de \gls{rlang} son:
\begin{itemize}
    \item \textbf{Análisis de datos}: \gls{rlang} fue escrito por estadísticos para estadísticos, por lo que está diseñado ante todo como un lenguaje para el análisis estadístico y de datos. Gran parte de la investigación de vanguardia en aprendizaje automático se realiza en \gls{rlang}, y cada semana se añaden paquetes a \gls{cran} que implementan estos nuevos métodos. Además, muchos modelos en \gls{rlang} pueden exportarse a otros lenguajes de programación como \gls{clang}, \gls{cpluspluslang}, \Gls{pythonlang}, etc.
    
    \item \textbf{Visualización de datos}: aunque el paquete básico de gráficos de \gls{rlang} es completo y potente, las bibliotecas adicionales como \gls{ggplot} y \gls{lattice} hacen que \gls{rlang} sea el lenguaje de referencia para los enfoques de visualización de datos más potentes.
\end{itemize}
%\newpage
%\cleardoublepage


\end{document}