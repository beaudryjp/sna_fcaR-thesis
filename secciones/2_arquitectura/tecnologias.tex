\documentclass[../../main.tex]{subfiles}

\begin{document}

\subsection{Arquitectura del Sistema}
    
\begin{comment}
    \begin{itemize}
        \item R: Lenguaje de programación bajo el cual esta escrito el proyecto.
        \item RStudio: Entorno de desarollo de R
        \item Overleaf: Herramienta online para la escritura de la memoria.
        \item MariaDB: Servidor de base de datos.
        \item phpMyAdmin: Gestor de servidor de base de datos
        \item Docker: Gestor de maquinas virtuales
    \end{itemize}
    
    Se utilizará Docker como gestor de maquinas virtuales esta herramienta para descargar una maquina virtual de un servidor de base de datos MariaDB\cite{doc11} y un gestor de base de datos phpMyAdmin\cite{doc12} ya preconfigurada.\\
\end{comment}



\subsubsection{Virtualización de Contenedores: Docker}

Docker\cite{doc10} es una plataforma de contenedores de código abierto. Docker permite a los desarrolladores empaquetar aplicaciones en contenedores -componentes ejecutables estandarizados que combinan el código fuente de la aplicación con todas las bibliotecas y dependencias del sistema operativo (SO) necesarias para ejecutar el código en cualquier entorno.

Aunque los desarrolladores pueden crear contenedores sin Docker, Docker hace que sea más fácil, sencillo y seguro crear, desplegar y gestionar contenedores. Es esencialmente un conjunto de herramientas que permite a los desarrolladores construir, desplegar, ejecutar, actualizar y detener contenedores utilizando comandos simples y automatización que ahorra trabajo.\\

Los contenedores ofrecen todas las ventajas de las máquinas virtuales, como el aislamiento de las aplicaciones, la escalabilidad rentable y la posibilidad de disponer de ellos.

Ventajas que proporciona Docker:
\begin{itemize}
    \item \textbf{Gestion de dependencias}: Permite gestionar las dependencias desde el sistema operativo hasta detalles como las versiones de los paquetes de R y Latex.
    
    \item \textbf{Reproducibilidad}: Asegura que los análisis realizados son reproducibles.
    
    \item \textbf{Portabilidad}: Existe una gran portabilidad permitiendo que se pueda trasladar un contenedor a otras maquinas.
    
    \item \textbf{Espacio en disco}: Ocupa muy poco espacio en comparación con las maquinas virtuales.
    
    \item \textbf{Creación automática de contenedores}: Docker puede construir automáticamente un contenedor basado en el código fuente de la aplicación.
    
    \item \textbf{Versionado de contenedores}: Docker puede hacer un seguimiento de las versiones de una imagen de contenedor, retroceder a versiones anteriores y rastrear quién construyó una versión y cómo.
    
    \item \textbf{Reutilización de contenedores}: Los contenedores existentes pueden utilizarse como imágenes base, básicamente como plantillas para construir nuevos contenedores.
    
    \item \textbf{Bibliotecas de contenedores compartidas}: Los desarrolladores pueden acceder a un registro de código abierto que contiene miles de contenedores aportados por los distintos usuarios.
\end{itemize}
\newpage
\cleardoublepage


\subsubsection{Bases de datos}
\paragraph{MariaDB}
MariaDB\cite{doc11} es un sistema de gestión de bases de datos relacionales de código abierto. Al igual que otras bases de datos relacionales, MariaDB almacena los datos en tablas formadas por filas y columnas. Los usuarios pueden definir, manipular, controlar y consultar los datos mediante el lenguaje de consulta estructurado SQL.

MariaDB esta basado en MySQL, por lo que ambas comparten muchas características y opciones de diseño.

Ventajas de MariaDB con respecto a MySQL:
\begin{itemize}
    \item MariaDB tiene 12 nuevos motores de almacenamiento mientras que MySQL tiene menos motores.
    \item MariaDB tiene un grupo de conexiones más grande que soporta hasta más de 200.000 conexiones, mientras que MySQL tiene un grupo de conexiones más pequeño.
    \item En MariaDB la replicación de los datos es más rápida mientras que en MySQL es más lenta.
    \item MariaDB es de código abierto, mientras que MySQL utiliza código propietario en su edición Enterprise.
    \item Comparativamente MariaDB es más rápido que MySQL.
\end{itemize}

\paragraph{phpMyAdmin}
phpMyAdmin\cite{doc12} es una herramienta de código abierto basada en PHP que permite administrar bases de datos MySQL y MariaDB en línea. Ofrece una interfaz web amigable al usuario bajo el cual se puede gestionar todo lo relativo a una base de datos y permite ejecutar consultas de Lenguaje de Consulta Estructurado (SQL).
\newpage
\cleardoublepage



\subsection{Lenguaje de Programación R}
R\cite{doc16} es un lenguaje de programación pensado para la computación estadística y la generación de gráficos. 

R ofrece una gran variedad de herramientas (paquetes) y técnicas las cuales permiten realizar cualquier tipo de análisis estadístico, aplicar algoritmos de inteligencia artificial o análisis de datos.

Los puntos puntos fuertes de R son:
\begin{itemize}
    \item \textbf{Análisis de datos}: R fue escrito por estadísticos para estadísticos, por lo que está diseñado ante todo como un lenguaje para el análisis estadístico y de datos. Gran parte de la investigación de vanguardia en aprendizaje automático se realiza en R, y cada semana se añaden paquetes a CRAN que implementan estos nuevos métodos. Además, muchos modelos en R pueden exportarse a otros lenguajes de programación como C, C++, Python, tensorflow, stan, etc.
    
    \item \textbf{Visualización de datos}: aunque el paquete básico de gráficos de R es completo y potente, las bibliotecas adicionales como ggplot2 y lattice hacen que R sea el lenguaje de referencia para los enfoques de visualización de datos más potentes.
\end{itemize}
\newpage
\cleardoublepage


\subsection{Herramientas y Entornos de Desarrollo}

\subsubsection{RStudio}
RStudio\cite{doc15} es un entorno de desarrollo integrado que permite interactuar con R más fácilmente. RStudio es en realidad un complemento del lenguaje de programación R, en el cual toma el software R y le añade una interfaz gráfica muy fácil de usar.

\subsubsection{Overleaf}
Overleaf\cite{doc17} es una herramienta de escritura y publicación colaborativa en línea de LaTeX que hace que todo el proceso de escritura, edición y publicación de documentos científicos sea mucho más rápido y sencillo.

\end{document}