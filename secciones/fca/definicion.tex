\documentclass[../../main.tex]{subfiles}

\begin{document}

\begin{teorema}
\textbf{Teorema Básico de Análisis de Conceptos Formales}. \\
El retículo de conceptos $\underline{\mathfrak{B}}(G, M, I)$ es un retículo completo. Para conjuntos arbitrarios de conceptos formales
\begin{center}
    $\{ (A_j, B_J) | j \in J \} \subseteq \underline{\mathfrak{B}}(G, M, I)$ \\
\end{center}
sus ínfimos y supremos vienen dados de la siguiente forma:
\begin{center}

    $\underset{j \in J}{\bigwedge} (A_j, B_j) = (\underset{j \in J}{\bigcap} A_j, (\underset{j \in J}{\bigcup} B_j)^{''}) $ , \\
    $\underset{j \in J}{\bigvee} (A_j, B_j) = ((\underset{j \in J}{\bigcup} A_j)^{''}, \underset{j \in J}{\bigcap} B_j) $ . \\
\end{center}
Un retículo completo $L$ es isomorfo al retículo $\underline{\mathfrak{B}}(G, M, I) \Longleftrightarrow$ si existe una función biyectiva $\gamma : G \rightarrow V$ y $\mu : M \rightarrow V$ tal que $\gamma(G)$ es supramemente denso en $\textbf{L}$, $\mu(M)$ es infimamente denso en $\textbf{L}$, y $gIm \Longleftrightarrow \gamma g \leq \mu m$ para todo $g \in G, m \in M$. En particular, $\textbf{L}$ es isomorfa a $\underline{\mathfrak{B}}(L, L, \leq)$.
\end{teorema}



\begin{definicion}
Un retículo de concepto de un contexto formal $\mathbb{K} = (G, M, I)$ es una estructura definida como $\langle \mathbb{K}, \leq \rangle$, dónde $\langle \mathbb{K}, \leq \rangle$ es una colección de todos los conceptos formales.
\end{definicion}

\begin{definicion}
Un \textbf{concepto formal} de un contexto formal $\mathbb{K} = (G, M, I)$ es un par $(A, B)$ con $A \subseteq G$, $B \subseteq M$, $A^{'} = B$ y $B^{'} = A$. Los conjuntos $A$ y $B$ se llaman extensión e intención del concepto formal $(A,B)$, respectivamente. La \textbf{relación subconcepto-superconcepto} viene dado por $(A_1, B_1) \leq (A_2, B_2) \Longleftrightarrow A_1 \subseteq A_2 (B_1 \subseteq B_2)$.
\end{definicion}


\begin{definicion}
El conjunto de todos los conceptos formales de un contexto $\mathbb{K}$ junto con relación de orden I forman un retículo completo, llamado el \textbf{retículo de conceptos} de $\mathbb{K}$ y se denota por $\underline{\mathfrak{B}}(\mathbb{K})$.
\end{definicion}

\end{document}