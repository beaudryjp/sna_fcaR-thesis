\documentclass[../../main.tex]{subfiles}

\begin{document}

\begin{teorema}
\textbf{Teorema Básico de Análisis de Conceptos Formales}. \\
El retículo de conceptos $\underline{\mathfrak{B}}(G, M, I)$ es un retículo completo. Para conjuntos arbitrarios de conceptos formales
\begin{center}
    $\{ (A_j, B_J) | j \in J \} \subseteq \underline{\mathfrak{B}}(G, M, I)$ \\
\end{center}
sus ínfimos y supremos vienen dados de la siguiente forma:
\begin{center}

    $\underset{j \in J}{\bigwedge} (A_j, B_j) = (\underset{j \in J}{\bigcap} A_j, (\underset{j \in J}{\bigcup} B_j)^{''}) $ , \\
    $\underset{j \in J}{\bigvee} (A_j, B_j) = ((\underset{j \in J}{\bigcup} A_j)^{''}, \underset{j \in J}{\bigcap} B_j) $ . \\
\end{center}
Un retículo completo $L$ es isomorfo al retículo $\underline{\mathfrak{B}}(G, M, I) \Longleftrightarrow$ si existe una función biyectiva $\gamma : G \rightarrow V$ y $\mu : M \rightarrow V$ tal que $\gamma(G)$ es supramemente denso en $\textbf{L}$, $\mu(M)$ es infimamente denso en $\textbf{L}$, y $gIm \Longleftrightarrow \gamma g \leq \mu m$ para todo $g \in G, m \in M$. En particular, $\textbf{L}$ es isomorfa a $\underline{\mathfrak{B}}(L, L, \leq)$.
\end{teorema}

\begin{definicion}
Una expresión de la forma $A\rightarrow B$ con $A, B\in M$ es una \textbf{implicación} en el contexto formal $K = (G, M, I)$ si $B\subseteq A''$, es decir, si cada objeto que posea los atributos del conjunto $A$, también posee los atributos del conjunto $B$.
\end{definicion}

\end{document}