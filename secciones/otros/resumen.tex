\documentclass[../../main.tex]{subfiles}

\begin{document}
%% begin abstract format
\makeatletter
\renewenvironment{abstract}{%
    \if@twocolumn
      \section*{Resumen \\}%
    \else %% <- here I've removed \small
    \begin{flushright}
        {\filleft\Huge\bfseries\fontsize{48pt}{12}\selectfont Resumen\vspace{\z@}}%  %% <- here I've added the format
        \end{flushright}
      \quotation
    \fi}
    {\if@twocolumn\else\endquotation\fi}
\makeatother
%% end abstract format

\begin{abstract}

En este trabajo fin de grado se presenta el desarrollo de una aplicación web que facilita la obtención de datos de ciertas redes sociales para realizar análisis de dichas redes con la finalidad de extraer información que se encuentra presente en el conjunto de datos que se obtiene. \\

La principal herramienta para conseguir este objetivo es hacer uso de Análisis de Conceptos Formales o AFC, la cual es una teoría matemática basada en la teoría de retículos y de la lógica para descubrir conocimiento de un conjunto de datos, esta herramienta es capaz de encontrar información de forma similar o incluso de una forma más eficiente que otras técnicas más conocidas como pueden ser las Reglas de Asociación. \\

Así mismo, se utilizará AFC con otras herramientas ya conocidas en la ciencia de datos para poder obtener el máximo grado de información posible de un conjunto de datos. \\

\noindent\textit{\textbf{Palabras claves:} Análisis de Conceptos Formales, Análisis de Redes Sociales, Minería de Datos, \gls{rlang} (Lenguaje de programación)}
\end{abstract}
\end{document}


