\documentclass[../../main.tex]{subfiles}

\begin{document}

La memoria de este proyecto se estructura de la siguiente manera:

\begin{itemize}
    \item \textbf{Introducción}: En este apartado se establece la motivación y los objetivos que este proyecto pretende aportar, así mismo se detalla la metodología usada.
    
    \item \textbf{Catálogo de Requisitos}: En este apartado se detallan los diferentes tipos de requisitos necesarios para el funcionamiento de la aplicación, así mismo los casos de uso para mostrar más en detalle el proceso general a seguir en la ejecución de la aplicación.
    
    \item \textbf{Arquitectura y Entorno Tecnológico}: En este capítulo viene detallado la arquitectura necesaria para que el proyecto funcione en cualquier ordenador independientemente del sistema operativo.
    
    \item \textbf{Análisis de Conceptos Formales}: Este capítulo viene descrito la base de proyecto en el cual se utiliza \gls{afc} para la extracción de conocimiento de datos, se define las definiciones matemáticas necesarias para poder describir el funcionamiento de esta estructura matemática. 
    
    \item \textbf{Análisis y extracción de la información}: Este capítulo describe el sistema encargado de realizar los procesos ETC (Extracción, Transformación y Carga) de los datos de las redes sociales para que se puedan manipular y tratar para su posterior uso. Así mismo también se describe como se analiza y extrae la información de las redes social usando \gls{afc}.
    
    \item \textbf{Implementación de la aplicación}: Este capítulo viene descrito la implementación de la aplicación web \Gls{shiny}, bajo el cual se mostrará al usuario una interfaz amigable para obtener datos de las redes sociales y mostrar la información de interés mediante visualizaciones.
    
    \item \textbf{Conclusiones y Líneas Futuras}: Este capítulo recoge las conclusiones finales del proyecto, donde se comenta se ofrece una valoración general sobre el proyecto y se define como mejorar el proyecto en sí.
    
    \item \textbf{Referencias}: En este capítulo viene incluido un listado con todas las fuentes bibliográficas consultadas en este trabajo fin de grado.
    
    \item \textbf{Apéndices}: En este capítulo se incluyen tres apéndices, donde viene incluido el manual de instalación y el manual de usuario.
    
\end{itemize}


\end{document}