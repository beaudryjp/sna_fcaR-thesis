\documentclass[../../main.tex]{subfiles}

\begin{document}

La memoria de este proyecto se estructura de la siguiente manera:

\begin{itemize}
    \item \textbf{Introducción}: En este apartado se establece la motivación y los objetivos que este proyecto pretende aportar, así mismo se detalla la metodología usada.
    
    \item \textbf{Arquitectura y Entorno Tecnológico}: En este capítulo viene detallado la arquitectura necesaria para que el proyecto funcione en cualquier ordenador independientemente del sistema operativo.
    
    \item \textbf{Análisis de Conceptos Formales}: Este capítulo viene descrito la base de proyecto en el cual se utiliza AFC para la extracción de conocimiento de datos, se define las definiciones matemáticas necesarias para poder describir el funcionamiento de esta estructura matemática. 
    
    \item \textbf{Diseño del sistema}: Este capitulo se describe el sistema encargado de realizar los procesos ETC (Extracción, Transformación y Carga) de los datos de las redes sociales para que puedan manipulados y tratados para su posterior uso.
    
    \item \textbf{Análisis y extracción de la información}: Este capítulo recoge como se analiza y extrae la información de las redes social usando AFC y otras técnicas de ciencia de datos.
    
    \item \textbf{Implementación de la aplicación}: Este capítulo viene descrito la implementación de la aplicación web Shiny, bajo el cual se mostrará al usuario una interfaz amigable para obtener datos de las redes sociales y mostrar la información de interés mediante visualizaciones.
    
    \item \textbf{Conclusiones y Líneas Futuras}: Este capítulo recoge las conclusiones finales del proyecto, dónde se comenta se ofrece una valoración general sobre el proyecto y se define como mejorar el proyecto en sí.
    
    \item \textbf{Referencias}: En este capítulo viene incluido un listado con todas las fuentes bibliográficas consultadas en este trabajo fin de grado.
    
\end{itemize}


\end{document}