\documentclass[../../main.tex]{subfiles}

\begin{document}

El objetivo principal de este proyecto es desarrollar un herramienta para el análisis de las redes sociales en la cual se permita extraer conocimiento oculto en los datos textuales almacenados de dichas redes. Esta aplicación servirá para facilitar el estudio de las redes sociales con lo cual permitirá ofrecer una ayuda a problemas relacionadas con estas redes, tales como es la segmentación de usuarios, la identificación de posibles clientes, análisis de mercados, estudios sociológicos, etc.
Para cumplir con este objetivo podemos desglosarlo de la siguiente forma:
\begin{itemize}
\item Diseño de procesos ETC (Extracción, Transformación y Carga), de manera que podamos obtener los datos deseados de las redes social, realizar la manipulación necesaria y importarlos en una base de datos relacional para su posterior uso.
\item Análisis de los datos obtenidos usando técnicas clásicas de la ciencia de datos y extracción de información haciendo uso de Análisis de Conceptos Formales\cite{fca}. 
\item Diseño de una Interfaz web dónde pueda obtener nuevos datos de las redes sociales en cuestión, que sea posible tratar con la información obtenida y que represente de forma visual los resultados obtenidos del análisis realizado.
\end{itemize}

\end{document}